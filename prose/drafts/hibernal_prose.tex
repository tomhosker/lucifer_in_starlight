How easily one lets blessedness slip away. For example: this morning in the hotel. We had been together all night (warmth to warmth, the hands of each in the softnesses of the other) and then I woke up and heard you already in the shower. It was still dark outside, only the streetlights glowing through the curtains; and also, even though I couldn't see or hear it, I remember in some sense I felt the snow fall. With its thin walls, it has all the intimacy of a private house, this place. (Not that I'm complaining for a moment; everyone knows I sleep like the dead.) The old man coughing away in the corridor, somebody's father, could well be your own father; the woman quietly being made love to in the room next door a housemate you've smouldered over for years. It's a two-edged thing this listening in on other people's lives but, inasmuch as it illuminates the bonds which join all of humanity, it makes me grateful for my own. I fumbled for my watch, looked at it, felt vaguely wronged. Then I went back to sleep.

When I woke up again you were standing over me, a murderous look in your eyes, and your short wet hair all on end like a devil's halo. I took the hint, showered and dressed quickly. Then you put down your book, put on your shoes, and we went down to breakfast.

[...]

Watching the sun set over the cathedral always gets me down. The symbolism is too obvious and too true: the decline of my country and the destruction its Christianity. And indeed the last and best of the Roman emperors cried out before his death:

\begin{quote}
    I can tell you this city once mastered the whole world: Armenia, Paphlagonia, Cappadocia, Georgia, Cilicia, Mesopotamia, Phoenicia, Aetolia, Achaea, Illyria, Egypt, Numidia, Spain and France. But now this inhuman sultan wants to violate her and make a slave out of the city of the world's desire. And our holy churches, where the Most Holy Trinity was worshipped, where the Holy Spirit was glorified in song, where angels were heard praising the incarnation and the divinity of God's Word, he wants to turn into shrines of his blasphemy, shrines of the mad and false prophet Muhammad, as well as into stables for his horses and camels.
\end{quote}

The day continues to fade; history proceeds according to God's immutable, inscrutable plan. And yet anything can happen: invincible armies crumple like tinfoil, whole nations spring up out of nowhere overnight. Wystan Auden, that great purveyor of homosexual love poetry, enjoyed, in middle age, a series of liaisons with a married woman\poeticmarginnote{Rhoda Jaff\'e} -- a fact that I ought to remind myself of when claiming to discern the pattern behind events. And I have you, my beloved, despite the darkness I can feel coming on.
