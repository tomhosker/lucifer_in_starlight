How easily one lets blessedness slip away. For example: this morning in the hotel. We had been together all night (warmth to warmth, the hands of each in the softnesses of the other) and then I woke up and heard you already in the shower. It was still dark outside, only the streetlights glowing through the curtains; and also, even though I couldn't see or hear it, I remember in some sense I felt the snow fall. With its thin walls, it has all the intimacy of a private house, this place. (Not that I'm complaining for a moment; everyone knows I sleep like the dead.) The old man coughing away in the corridor, somebody's father, could well be your own father; the woman quietly being made love to in the room next door a housemate you've smouldered over for years. It's a two-edged thing this listening in on other people's lives but, inasmuch as it illuminates the bonds which join all of humanity, it makes me grateful for my own. I fumbled for my watch, looked at it, felt vaguely wronged. Then I went back to sleep.

When I woke up again you were standing over me, a murderous look in your eyes, and your short wet hair all on end like a devil's halo. I took the hint, showered and dressed quickly. Then you put down your book, put on your shoes, and we went downstairs.

It's a fair walk up to the cathedral: not far, but steep. And I always enjoy it, even in foul weather. Today was far from foul: one of those clear cold days when the mind feels intensely awake: cornflower blue skies and that marvellous crunch of untouched snow underfoot. It was a few hot minutes climbing the old stone steps winding up from the river, and there she was: the handsomest of our English cathedrals.

It was lovely and warm inside; I explained to you why the climate inside these big stone buildings is always the opposite of the weather outdoors. In a side chapel, a priest and a handful of congregants had just finished celebrating the eucharist, the altar still decked out in Epiphany white and gold. I lit two candles and prayed for a few moments. I thought about my grandfather, who had died unhappily and not exactly in a state of grace; I also thought about that man who joined my last ship a few months before I did, who hanged himself from the deckhead, next to one of the tumble driers. Then I made my umpteenth pilgrimage to the shrine of Saint ---.

Bernie was stood waiting for us when we got outside. He was still wearing his naval greatcoat, and I teased him that he looked like he'd just got back from marshalling lifeboats off of the \textit{Titanic}.

[...]

Watching the sun set over the cathedral always gets me down. The symbolism is too obvious and too true: my country's dwindling fortunes, and the twilight of her Christianity. How can I live my country's failures when I can barely come to terms with my own? Has any nation ever come back from this kind of malaise? Some, maybe. Many were rubbed out in an instant; most withered over a number of years. Indeed the last and best of the Roman emperors told his soldiers before he died:

\begin{quote}
    I can tell you this city once mastered the whole world: Armenia, Paphlagonia, Cappadocia, Georgia, Cilicia, Mesopotamia, Phoenicia, Aetolia, Achaea, Illyria, Egypt, Numidia, Spain and France. But now this inhuman sultan wants to violate her and make a slave out of the city of the world's desire. And our holy churches, where the Most Holy Trinity was worshipped, where the Holy Spirit was glorified in song, where angels were heard praising the incarnation and the divinity of God's Word, he wants to turn into shrines of his blasphemy, shrines of the mad and false prophet Muhammad, as well as into stables for his horses and camels.
\end{quote}

Time continues to flow at a uniform rate; history proceeds according to God's immutable, inscrutable plan. And yet anything can happen: invincible armies crumple like tinfoil, whole nations spring up overnight. Wystan Auden, that great purveyor of homosexual love poetry, enjoyed, in middle age, a series of liaisons with a married woman\poeticmarginnote{Rhoda Jaff\'e} -- a fact that I ought to remind myself of when claiming to discern the pattern behind events. A braver man's courage would soften his hardships; and I have you, my beloved, despite this darkness I can feel coming on.
