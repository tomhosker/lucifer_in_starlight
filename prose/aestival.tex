\noindent \textsc{Juan was} very good with women. If you have never been -- as I was when I knew him -- a sixteen year old male, free of my parents for the first time, in the intense heat of a Roman summer, you might not appreciate how close to godhood a talent with women can make a man in the eyes of his companions. But, for a young man, being good with women is not like being good with cars or horses or differential equations; those are all a means to an end, but the enjoyment of women is the summit of human existence.

Juan enjoyed plenty of women, but it would be dishonest of me if I gave you the impression that he was ever callous or possessive towards his lovers. I never saw him look at a woman as if she was his Juliet, or as if she was just another entry in a ledger of conquests. Perhaps that was his secret.

Juan was good with women, but you have probably already guessed that he stirred up a different kind of admiration in my own heart too. Just like in Solomon's dream\poeticmarginnote{I Kgs 3.5-14} -- where God poured out all his other blessings because the king had the sense to ask for the one thing that was truly valuable -- Juan's talent with women seemed to be complemented by every other conceivable advantage. He was young, and I hardly need to tell you he was handsome -- but not so handsome as to obscure his generous good nature. He had a sharp mind and came from a wealthy family; when I knew him, he was taking some sort of extended holiday from an advanced degree. He was Costa Rican, but without that inflated national pride which citizens of younger countries sometimes carry within themselves; I remember one evening when we were smoking hash on a balcony, a mutual friend peeled off a ``Product of Costa Rica'' sticker from a mango and stuck it on his teeshirt, and he laughed heartily. And he also happened to possess a straightforwardly reassuring religious faith; indeed, we used to take the sacrament together at Santa Maria Maggiore, and watch the blossoms fluttering down in memory of the miraculous August snow.

\prosesep

One evening -- I remember it was a Friday -- we went for a night out in Ostia, and it happened like this.\poeticmarginnote{Jn 21.1} There's a train line from Basilica San Paolo to the centre of Ostia, and we bought two litres of vodka and a large carton of orange juice from a shop near the station. We took the last train out; I think in those days it left around eleven. The journey takes half an hour, and, because we had no third vessel in which to mix the juice and the alcohol, I would take a swig from the carton and Juan from the bottle, after which he would decant a glug or so from his vessel into mine. And I was astonished by the nonchalance with which he drank, as if he was imbibing nothing more caustic than lukewarm camomile tea.

In Ostia, there was a nightclub on the beach, but we rejoined Juan's friends about a hundred yards further down the shoreline, where they had built a small fire. There was a short, muscular Brazilian man, who wanted to be a boxer. There was a young Mexican, whose father had recently bought him a tiny bar in Rome proper (called La Radio) and his Italian girlfriend, who, over the course of the evening, made a number of rather funny comparisons between her boyfriend and Speedy Gonzales, which he took with very good grace. And there were also three or four young Italian men, whose names and faces I have forgotten.

Juan was friendly with the bouncers at the nightclub, so that we could come and go between the club and the campfire as we pleased. By one o'clock, Juan and myself were bored of vodka orange, and so headed for the bar for a proper drink. It was busy, and there were only a couple of bartenders; we were clearly in for a wait. But that was no matter; it gave us a chance to mingle with the girls. I got talking to a young Italian lady; her name was Vittoria. She was older than me, but only by a few months. And she was lovely to look at. She had the features one might expect from an Italian; skin half a shade darker than my own, deep brown eyes, wavy black hair. She also had exceptionally rosy cheeks which glowed a little when she smiled, even though -- except perhaps for a touch of mascara -- she was not wearing any makeup. Juan, of course, was doing his thing; I saw him talking to a girl at the other end of the enclosure. To the best of my recollection, she looked about fifteen, with light brown hair and watery blue eyes. But she stood out because of the refinement of her clothing; she wore a pearl grey cocktail dress -- a bit short, but very elegant -- paired with diamond ear studs and a pair of black stilettos. When I finally got to the bar, I asked for a \textit{gin tonica} and a vodka lime for Vittoria. I gestured to Juan asking if he would like me to get him anything, but he gestured back that he was fine.

As we walked back down the beach, I noticed some girls darting in and out of the waves, playing in the water like otters.

`\textit{Sono...?}' I asked my companion. \hspace{0.5in} {\footnotesize Are they...?}

`\textit{Completamente,}' she replied. \hspace{0.5in} {\footnotesize Completely.}

And I cursed my shortsightedness. All I could see were pink outlines in the throbbing blackness.

Vittoria had come with her brother, but she did not mind being separated from him for a bit; neither twin was overprotective like that. So we settled back down by the campfire, and spent the next few hours drinking and talking. I tried my hardest to absorb her \textit{italianit\`a}, and she put up with my weak grasp of her language most commendably.

Juan reappeared an hour before dawn, carrying a girl on his back as if he was giving her a piggyback. As he approached the fire, he laid her down for a moment on the sand. It was one of the American girls who was living in the house with us. (Her name was Heidi.) She looked nine tenths unconscious. I began to get up.

`No,' Juan said. `She's okay. I found her in the club, slumped over one of the tables. She just had a bit too much. She'll be all right.'

We took the first train from Ostia back to Basilica San Paolo, which left at about five o'clock. Vittoria was sat next to me; I finally worked up the courage to put my arm round her shoulders; and we cuddled chastely in the pre-dawn twilight for the half hour journey back into the city. We parted ways forever outside the station, and Juan and myself looked for a bus to take Heidi's mortal remains back to our house near the Ponte Casilino.

The bus was practically empty, so we took the back row. We propped Heidi against the window, and she soon passed out with her face against the glass. Naturally, Juan and myself began to talk. What was he doing while I was talking to Vittoria by the fire? Where had he been?

`I was with a girl.'

`The one I saw you with by the bar?'

`Yes.'

`And you...?'

`Yes.'

`Where?'

`We walked a bit further down the coast. I think perhaps for half a mile? There was a hotel. It was deserted. And there were plenty of loungers.'

`That's incredible.'

`That's not even the best part.' He smiled to himself. `I think she was a virgin. She was nervous and --' he pulled down his jeans to show me his underwear; on the grey cotton, there was a trail of black splodges near the crotch -- `she bled on me a bit. Anyway, she won't forget me.'

Just as he finished speaking, the bus turned and passed under the Porta Onoriana, a trio of rose-gold coloured arches, a stone's throw from the main part of the Porta Maggiore, the remains of a fortification put up in the last years of the Western Roman Empire. It was a cloudless day, and, because it was so early, the sun was still low enough to shine directly through the archways, pouring its marvellous burgundy light down the whole length of the Via Casilina.\poeticmarginnote{I Pet 2.9} Juan and I and glanced at each other. Perhaps there was even a chuckle or a sharp intake of breath. But no words were necessary to acknowledge the moment. We passed the rest of the journey in a comfortable silence.

Heidi perked up once we were home. Juan made cheese and prosciutto omelettes for the three of us. He was, of course, an excellent cook. Then I went to my room and slept until well into the afternoon.

\prosesep

Years later, my beloved, in a stonebuilt house in England, a few minutes after I had relieved you of your own virginity, I asked you if I could get you anything, and you asked for hot chocolate. I went down to the kitchen, put the kettle on, and looked through the window at the night's sky. It was just after three o'clock in the morning of the twenty-fifth of December; but, for all the distance in time, climate and latitude, I could not help but think back to that morning on the Via Casilina.
