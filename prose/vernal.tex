\noindent \textsc{I remember}, when I was seventeen, kissing a girl in a hotel restaurant in Sorrento. It was in the early hours of the morning; apart from us and another teenage couple, the place was deserted. I remember that she was wearing her pyjamas and an elegant gold pendant and a heavy, sweet perfume. She was Italian but she was a tourist like me, a Venetian; the south is really another country. The only illumination came from the lights in the street outside. Even when one puts to one side the lack of choosiness of drunk seventeen year old males, she was superlatively beautiful. She pressed her face against me and kissed me, and I put my left hand on the bare skin of her hip. She continued to kiss me, and I moved that hand further under her pyjamas. I remember the nipple of her right breast grazed the ring finger of my left hand, the first time I had touched that part of a woman since I took my mother's milk.

An\poeticmarginnote{Heraclitus v Wittgenstein} otherwise obscure philosopher wrote that one cannot step into the same river twice -- because it's not the same river and it's not the same man -- but I am strongly inclined towards the opposite point of view. All the molecules in my body -- lips, eyes and hands -- have been replaced several times since that evening in Italy, and yet these hands, in the only sense which has any practical significance, are the same hands that touched that much-missed Italian. The river is not the same thing as the water flowing through it.

\prosesep

Suppose I cut myself one afternoon while removing rust with a knife. The wound runs almost the entire length of the index finger of my left hand. If it's deep enough, it will leave an impressive scar, which, though it may fade, will be with me for the rest of my life. As with every other part of my body, all the molecules in the scar will be replaced over time. But the scar itself will remain, and if, years later, I were to point to it and say that this is where I cut myself, I would be telling the truth.

\prosesep

I have this recurring dream about my grandmother. I am standing in the living room of her old house, and she lets herself in through the front door. Where has she been? Had she not died? `Yes,' she says. `But I was only dead for a few months. That doesn't count.' (In fact, she has been dead many years, but this never occurs to me in my dream.) In life, she was not any kind of scholar, nor would she ever have become one even if she had been born into great privilege. But in my dream she utters sentences of the most dazzling sagacity, the logic of which inevitably falls apart in the morning. And the same unselfconscious kindness is there, the same quietly nurturing presence as I remember. If I try to ask her about what it feels like to be dead, she dissolves into another dream.

\prosesep

If we live again, will it be as a physical creature of some kind, or will it be as an intangible spirit? I am inclined to believe that any afterlife worthy of the name would have to take a corporeal and physical form, although without necessarily conforming to the physics of this present reality; indeed I have difficulty believing that an intangible existence is even possible in the first place. But, if my body is so indispensible to my living again, how could anyone guarantee that this new body was in fact my body, and not some kind of living wetsuit that my consciousness had stretched over itself? Does this problem actually point towards the impossibility of an afterlife?

\prosesep

When I try to imagine an eternity worth inhabiting, my mind turns inexorably towards the resurrection appearances described in the gospels. I can think of three reasons for this. (1) My upbringing was such that I instinctively think about death and the hereafter within the thought-world of Christianity. (2) The gospels seem to answer some of the philosophical objections to an afterlife -- continuity, identity, boredom, etc -- in a robust fashion. (3) The accounts of Jesus' resurrection have a compelling strangeness, not unlike an etching of an impossible object, which draw in the mind's eye long after being heard.

Assume for a moment that Jesus really did rise from the dead. Did he, on Sunday morning, inhabit the same body that the soldiers had nailed to the cross on Friday lunchtime, or was he a ghost of some sort? The gospels stress the physicality of the risen Christ. In Matthew, the women cling to his feet. In John, he famously invites one of his disciples -- who probably earned his nickname Thomas, a Greek speaker's mangling of the Aramaic word for twin, due to a striking physical resemblance to Jesus -- to put his fingers in the five wounds of the crucifixion. Luke tackles this question quite explicitly; here Jesus declares to his apostles, \textit{Touch me and see. A spirit does not have skin and bones as you see that I have.} But this body, however tangible, is not like any normal body. The risen Jesus walks through walls, appearing and disappearing at will, and barely stays in one locality for more than the length of a conversation. In the famous story of the road to Emmaus, he talks with two of his followers for what must have been several hours, but they do not recognise him -- except when, at the end of their journey, he breaks bread in a distinctive way, at which point he vanishes in plain sight. The risen Christ seems both to have re-entered his mortal body and, simultaneously, to have transmuted that same body into a new kind of matter.

\poeticmarginnote{I Cor 15.35-37}\textit{How are the dead raised? With what kind of body do they come? You foolish man. What you sow does not come to life unless it dies, and what you sow is not the body which is to be, but a bare kernel.}

One other thing: the gospels emphasise the historicity of Jesus' resurrection, Luke especially. But this only rediscovers the paradox at the heart of the resurrection from another point of view. History\poeticmarginnote{Walsh} has to do with finding similarities and differences in order to understand what happened, but, if the descriptions of the resurrection are literally true, there is nothing with which to make a comparison.

\prosesep

I\poeticmarginnote{Berryman} have no idea whether we live again. It hardly seems likely from either the scientific or philosophical point of view. \poeticmarginnote{Mt 19.26}But all things are possible with God, and, when I try to imagine an eternity worth inhabiting, what I see are the resurrection appearances described in the gospels.
