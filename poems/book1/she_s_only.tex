\settowidth{\versewidth}{To her tapes, and sometimes smile, belle of Coach F.}
\begin{verse}[\versewidth]
She's only nine tenths beautiful,\poeticmarginnote{Blind Girl on a Train}\\*
\vin And yet we men are standing reverently around,\\
\vin \vin Watching her listening\\*
To her songs, and sometimes smile, belle of Coach F.\\!

Her hair is tangled \& black like black-bryony,\\*
\vin Ringletting down to her seat, beneath which\\
\vin \vin A violin case\\*
Edged with red velvet is tucked between her calves.\\!

And those cheeks, as tender as meltwater\\*
\vin And as clear, that clear voice purling\\
\vin \vin More rosily than mulled wine,\\*
That six hand\poeticmarginnote{hand = 4 in} waist no soul has yet handled.\\!

But the light fails, lovers meet and journeys end.\marginnote{\footnotesize \textit{As You Like It} II.5}\\*
\vin And she alights, to be met by her first sweetheart,\\
\vin \vin Whose head she kisses\\*
As if the world, and not her, had gone blind:\\!

And nothing matters more than a girl,\\*
\vin Of a velvety, late august early evening,\\
\vin \vin With her fingers\\*
In the lapels of her lover's jacket.
\end{verse}

\bigskip

\begin{minipage}{\textwidth}
    \footnotetext{The poet may have had a passage from Charles Williams's \textit{Witchcraft}, quoted by Prof Auden in \textit{The Dyer's Hand}, in the back of his mind when these lines came to him, namely: `One is aware that a phenomenon, being wholly itself, is laden with universal meaning. A hand lighting a cigarette is the explanation of everything; a foot stepping from the train is the rock of all existence... Two light dancing steps by a girl appear to be what all the [Scholastic philosophers] were trying to express... but two quiet steps by an old man seem like the very speech of hell. Or the other way round.'}
\end{minipage}
