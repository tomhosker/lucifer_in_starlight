\textsc{I remember}, when I was seventeen, kissing a girl in a hotel restaurant in Sorrento. It was in the early hours of the morning; apart from us and another teenage couple, the place was deserted. I remember that she was wearing her pyjamas and an elegant gold pendant and a heavy, sweet perfume. She was Italian but she was a tourist like me, a Venetian; the south is really another country. The only illumination came from the lights in the street outside. Even when one puts to one side the lack of choosiness of seventeen year old boys, she was superlatively beautiful. She pressed her face against me and kissed me, and I put my left hand on the bare skin of her hip. She continued to kiss me, and I moved that hand further under her pyjamas. I remember the nipple of her right breast grazed the ring finger of my left hand, the first time I had touched that part of a woman since I took my mother's milk.

An\poeticmarginnote{Heraclitus v Wittgenstein} otherwise obscure philosopher wrote that one cannot step into the same river twice, but I am strongly inclined towards the opposite point of view. All the molecules in my body -- lips, eyes and hands -- have been replaced several times since that evening in Italy, and yet these hands, in the only sense which has any practical significance, are the same hands that touched that much-missed Italian. The river is not the same thing as the water flowing through it.

Suppose I cut myself one afternoon while removing rust with a knife. The wound runs almost the entire length of the index finger of my left hand. If it's deep enough, it will leave an impressive scar, which, though it may fade, will be with me for the rest of my life. As with every other part of my body, all the molecules in the scar will be replaced over time. But the scar itself will remain, and if, years later, I were to point to it and say that this is where I cut myself, I would be telling the truth.

\bigskip
\centerline{\vbox{\hrule width 2in}}
\bigskip
\bigskip

I have this recurring dream about my grandmother. I am standing in the living room of her old house, and she lets herself in through the front door. Where has she been? Had she not died? `Yes,' she says. `But I was only dead for a few months. That doesn't count.' (In fact, she has been dead many years, but this never occurs to me in my dream.) In life, she was not any kind of scholar, nor would she ever have become one even if circumstances had made such a thing possible. But in my dream she utters sentences of the most dazzling sagacity, the logic of which inevitably falls apart in the morning. And the same unselfconscious kindness is there, the same quietly nurturing presence as I remember. If I try to ask her about what it feels like to be dead, she dissolves into another dream.

\bigskip
\centerline{\vbox{\hrule width 2in}}
\bigskip
\bigskip

If there is a life after this one, will it be as a physical creature of some kind, or will it be as an intangible spirit? I am inclined to believe that any afterlife worthy of the name would have to take a corporeal and physical form, although without necessarily conforming to the physics of this present reality; indeed I doubt whether an intangible existence is even possible in the first place. But, if my body is so indispensible to my living again, how would I know that this new body was in fact my body, and not some kind of living wetsuit that my incorporeal essence stretched over itself? Does this problem actually point towards the impossibility of an afterlife?
