\textsc{How easily} one lets blessedness slip away. For example: this morning in the hotel. We had been together all night (warmth to warmth, the hands of each in the softnesses of the other) and then I woke up and heard you already in the shower. It was still dark outside, only the streetlights glowing through the curtains; and also, even though I couldn't see or hear it, I remember in some sense I could feel the snow fall. I fumbled for my phone, looked at it, felt vaguely wronged. Then I went back to sleep.

When I woke up again you were standing over me, a murderous look in your eyes, and your wet hair all on end like a devil's halo. I took the hint, showered and dressed quickly. Then you put down your book, put on your shoes, and we went downstairs.

It's a fair walk up to the cathedral: not far, but steep. And I always enjoy it, even in foul weather. Today was far from foul: one of those clear cold days when the mind feels intensely awake: cornflower blue skies and that marvellous crunch of untouched snow underfoot. It was a few hot minutes climbing the stone steps winding up from the river, and there she was: the handsomest of our English cathedrals.

It was lovely and warm inside; I explained to you why the climate inside these big stone buildings is always the opposite of the weather outdoors. In a side chapel, a priest and a handful of congregants had just finished celebrating the eucharist, the altar still decked out in Epiphany white and gold. I lit two candles and prayed for a few moments. I thought about my grandfather, who had died unhappily and not exactly in a state of grace. I also thought about that poor soul from my seagoing days. He had joined a ship in Portland -- a few months before I joined that ship myself, my last deployment but one -- and had hanged himself from the deckhead, next to one of the tumble driers. Then I made my umpteenth pilgrimage to the shrine of St ---.

Bernie was stood waiting for us when we got outside. He was still wearing his naval greatcoat, and I teased him that he looked like he'd just got back from marshalling lifeboats off of the \textit{Titanic}. He said he knew a good place, down one of the alleyways off Silver Street, opposite the Seven Altars Cafe\poeticmarginnote{Nine Altars~| The Shakespeare}: a pub called The Marlowe. So we followed him there, a five minute walk. It wasn't much to look at; to tell you the truth it looked like a real spit and sawdust kind of place. But the beer was mostly local, and the all day breakfast came with a Cumberland sausage of a noticeably high end stamp -- not to mention that the taproom had a good view of the river and the snowy countryside beyond it. It was a good effort on Bernie's part; it reminded me why I liked him.

Bernie and myself got to talking about fornication. This was comfortable territory; we had sailed together a couple of times, and each revelled in the other's extravagant bad taste. To your credit, you were content just to listen, only butting in with a witticism of your own very occasionally; you had long since internalised the scriptural proverb about the imagination of a man's heart being evil from his youth.\poeticmarginnote{Gen 8.21} Time went by at an alarming rate, as it tends to with an old friend and a functionally bottomless supply of good beer. Then at about half past two in the afternoon we got to talking about which was worse, very hot weather or very cold. You remarked to Bernie that his opinion was the expert one; he must have gone through plenty of both in Afghanistan. He was silent for a moment. Then he said softly, `Yes. Yes, I must have. But the heat was worse.' And he smiled at you with uncharacteristic meekness.

I winked at you with the eye Bernie couldn't see from his side of the table. We both -- Bernie and myself -- disliked talking about our service in detail, albeit for different reasons. `It's not that I don't want to talk about it,' he once told me; `it's that no one really wants to listen. Everyone has these preconceived ideas of what it must have been like.' In the beginning, he was troubled by more abstract considerations: Afghanistan's reputation as a graveyard for foreign armies, the transparent lunacy of sending British sailors to resolve political crises in Central Asia. Later on, he found more immediate concerns. I never pressed him, but he alluded several times to patrolling the outskirts of Camp Bastion; the other side had got very good at firing rockets so as to explode over the heads of our guys, often with lethal effect. But Bernie had so much he could have remembered with pride. I, on the other hand, was never sent to fight; the closest I ever got to a warzone was admiring the mountains of Yemen through my cabin porthole. And my career had suffered from, one might say, several other infelicities. That was what lay behind my own reticence.

Given the lull in the conversation, I excused myself to look round a bookshop. It was just round the corner, a secondhand place, and if I didn't go now, I told yourself and Bernie, it would shut before I had a chance to pay it a visit. I stayed an hour -- I would have stayed longer but for you -- yet the smell of the venerable paper and the craftsmanship of the bindings always brightened my mood. I left with an Edwardian volume of translations from Tibullus. When I got back to the pub, I managed to slip in without you noticing me; I stopped for a few moments and watched you both chatting excitedly, my ears still burning from the cold. I had no fear of either of you ever betraying me, but sometimes I wondered if you hadn't chosen the wrong man. We had another drink; I must have had at least seven pints while we were talking but, spaced out over an afternoon, and with food, I was only just then beginning to feel drunk. Then we all left together, returning the same way as we had come.

Watching the sun set over the cathedral always gets me down. The symbolism is too obvious and too true: my country's dwindling fortunes, and the twilight of her Christianity. How can I live with my country's failures when I can barely come to terms with my own? Has any nation ever come back from such gangrenous loculations? A couple, maybe; many more were stubbed out in an instant; most fell apart over a few generations. Indeed the last and best of the Roman emperors told his soldiers shortly before the bloodletting which finished off his dominion:

\begin{quote}
    I can tell you this city once mastered the whole world: Armenia, Achaea, Macedonia, Italy, Georgia, Mesopotamia, Syria, Spain, Illyria, Egypt, England and France. But now this inhuman sultan wants to violate her and make a slave out of the city of the world's desire. And our holy churches, where the Most Holy Trinity was worshipped, where the Holy Spirit was glorified in song, where angels were heard praising the incarnation and the divinity of God's Word, he wants to turn into shrines of his blasphemy, shrines of the mad and false prophet Muhammad, as well as into stables for his horses and camels.
\end{quote}

Time continues to flow at a uniform rate; history proceeds according to God's immutable, inscrutable plan. And yet anything can happen: invincible armies crumple like tinfoil, whole nations spring up overnight. Wystan Auden, that great purveyor of homosexual love poetry, enjoyed, in middle age, a series of liaisons with a married woman\poeticmarginnote{Rhoda Jaff\'e} -- a fact that I ought to remind myself of when claiming to discern the pattern behind events. If only I had more courage, things might have worked out; but I have you, my beloved, despite this darkness I can feel coming on.
